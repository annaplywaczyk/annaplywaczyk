\documentclass{book}
\usepackage[a4paper,top=2.5cm,bottom=2.5cm,left=2.5cm,right=2.5cm]{geometry}
\usepackage{makeidx}
\usepackage{natbib}
\usepackage{graphicx}
\usepackage{multicol}
\usepackage{float}
\usepackage{listings}
\usepackage{color}
\usepackage{ifthen}
\usepackage[table]{xcolor}
\usepackage{textcomp}
\usepackage{alltt}
\usepackage[utf8]{inputenc}
\usepackage{polski}
\usepackage[T1]{fontenc}

\usepackage{mathptmx}
\usepackage[scaled=.90]{helvet}
\usepackage{courier}
\usepackage{sectsty}
\usepackage{amssymb}
\usepackage[titles]{tocloft}
\usepackage{doxygen}
\lstset{language=C++,inputencoding=utf8,basicstyle=\footnotesize,breaklines=true,breakatwhitespace=true,tabsize=8,numbers=left }
\makeindex
\setcounter{tocdepth}{3}
\renewcommand{\footrulewidth}{0.4pt}
\renewcommand{\familydefault}{\sfdefault}
\hfuzz=15pt
\setlength{\emergencystretch}{15pt}
\hbadness=750
\tolerance=750
\begin{document}
\begin{titlepage}
\vspace*{7cm}
\begin{center}
{\Large My Project }\\
\vspace*{1cm}
{\large Wygenerowano przez Doxygen 1.8.1.2}\\
\vspace*{0.5cm}
{\small N, 2 mar 2014 21:47:19}\\
\end{center}
\end{titlepage}
\clearemptydoublepage
\pagenumbering{roman}
\tableofcontents
\clearemptydoublepage
\pagenumbering{arabic}
\chapter{Program wyliczajacy czas wykonywanego alorytmu}
\label{index}\begin{DoxyAuthor}{Autor}
Anna Plywaczyk, 200340 
\end{DoxyAuthor}
\begin{DoxyDate}{Data}
02.\-03.\-2014 
\end{DoxyDate}
\begin{DoxyVersion}{Wersja}
0,1
\end{DoxyVersion}
Aplikacja porozumiewa sie z uzytkowniem. Prosi o podanie nazwy pliku, na ktorym ma byc wykonany algorytm. W programie zostanie wlaczony stoper, ktorym zmierzymy czas w jakim algorytm zostanie wykonany. Pomiar czasu zostanie wykonany kilkakrotnie i na ekran zostanie wypisana wartosc srednia. Aplikacja ponownie poprosi o podanie nazwy pliku, ktory ma zostac porownany z plikiem pomnozonym przez 2. Jezeli wektory beda jednakowe program zworoci komunikat o tym iz mnozenie zostalo wykonane poprawnie w przeciwnym wypadku zostanie zwrocony komunikat o blednym wykonaniu algorytmu. 
\chapter{Indeks plików}
\section{Lista plików}
Tutaj znajduje się lista wszystkich plików z ich krótkimi opisami\-:\begin{DoxyCompactList}
\item\contentsline{section}{{\bf zad1.\-cpp} \\*Program wczytuj�cy plik, wykonuje algorytm na pliku i por�wnuje z prawid�owym plikiem. Program napisany jest w jednym pliku z u�yciem funkcji }{\pageref{zad1_8cpp}}{}
\end{DoxyCompactList}

\chapter{Dokumentacja plików}
\section{Dokumentacja pliku strona.\-dox}
\label{strona_8dox}\index{strona.\-dox@{strona.\-dox}}

\section{Dokumentacja pliku zad1.\-cpp}
\label{zad1_8cpp}\index{zad1.\-cpp@{zad1.\-cpp}}


Program wczytuj�cy plik, wykonuje algorytm na pliku i por�wnuje z prawid�owym plikiem. Program napisany jest w jednym pliku z u�yciem funkcji.  


{\ttfamily \#include $<$iostream$>$}\\*
{\ttfamily \#include $<$fstream$>$}\\*
{\ttfamily \#include $<$vector$>$}\\*
{\ttfamily \#include $<$string$>$}\\*
{\ttfamily \#include $<$cmath$>$}\\*
{\ttfamily \#include $<$windows.\-h$>$}\\*
Wykres zależności załączania dla zad1.\-cpp\-:
\subsection*{Funkcje}
\begin{DoxyCompactItemize}
\item 
void {\bf wczytywanie} (const string \&tekst, vector$<$ int $>$ \&dane)
\begin{DoxyCompactList}\small\item\em Funkcja wczytuj�ca dane do wektora z pliku. Funkcja otwiera plik zadany przez u�ytkownika, sprawdza czy plik zosta� otwarty, je�eli zosta� otwarty wczytana jest liczba elementow pliku, nastepnie wczytywane sa wszystkie liczby do wektora z danymi. \end{DoxyCompactList}\item 
void {\bf wypisz} (vector$<$ int $>$ \&dane)
\begin{DoxyCompactList}\small\item\em Funkcja wypisuj�ca na ekran wszystkie elementy wektora. Funkcja wypisuje na ekran wsyztskie elemnty z pliku na ekran w postaci wektora w kolumnie. \end{DoxyCompactList}\item 
void {\bf algorytm} (vector$<$ int $>$ \&dane)
\begin{DoxyCompactList}\small\item\em Funkcja wykonuj�ca zadany algorytm na wektorze. Funkcja wykonuje zadany algorytm na wektorze wejsciowym. W naszym przypadku wektor pomno�ony jest przez sta�� liczb� 2. \end{DoxyCompactList}\item 
void {\bf porownanie} (vector$<$ int $>$ \&dane, vector$<$ int $>$ \&poprawne)
\begin{DoxyCompactList}\small\item\em Funkcja por�wnuj�ca dwa wektory. Funkcja por�wnuje dwa wektory, sprawdza czy wszystkie elemnty s� ze sob� r�wne. W ka�dym wypadku zwraca nam komunikat albo �e wektory poddane analizie s� jednakowe b�d� b��dne. Do funkcji poddawna s� wektory jeden po wprowadzeniu z pliku i wykonaniu algorytmu oraz drugi, kt�ry jest ma by� sprawdzeniem czy algorytm zosta� prawid�owo wykonany. \end{DoxyCompactList}\item 
L\-A\-R\-G\-E\-\_\-\-I\-N\-T\-E\-G\-E\-R {\bf wylacz\-Stoper} ()
\begin{DoxyCompactList}\small\item\em Funkcja zapamietuj�ca czas poczatkowy. Funkcja nale��ca do biblioteki \char`\"{}windows.\-h\char`\"{}, stoper zostaje w��czony, aby zosta� zmierzony czas wykonania algorytmu w programie. Funkcja nale��ca do funkcji bool Query\-Performance\-Counter(\-\_\-out L\-A\-R\-G\-E\-\_\-\-I\-N\-T\-E\-G\-E\-R $\ast$\-Ip\-Performance\-Count), funkcja ta zwraca wartosc niezerowa je�eli w�aczenie zako�czy si� sukcesem, natomiast w przeciwnym wypadku zostanie wyrzucony b��d i zwr�ci wartosc 0. Dla komputer�w multiprocesorowych nie ma znaczenia, kt�ry jest u�ywany. \end{DoxyCompactList}\item 
L\-A\-R\-G\-E\-\_\-\-I\-N\-T\-E\-G\-E\-R {\bf wlacz\-Stoper} ()
\begin{DoxyCompactList}\small\item\em Funkcja zapamietuj�ca czas ko�cowy. Funkcja nale��ca do biblioteki \char`\"{}windows.\-h\char`\"{}, stoper zostaje wy��czony, aby zosta� zmierzony czas wykonania algorytmu w programie, poprzez odj�cie czasu pocz�tkowego od czasu ko�cowego. Funkcja nale��ca do funkcji bool Query\-Performance\-Counter(\-\_\-out L\-A\-R\-G\-E\-\_\-\-I\-N\-T\-E\-G\-E\-R $\ast$\-Ip\-Performance\-Count), funkcja ta zwraca wartosc niezerowa je�eli w�aczenie zako�czy si� sukcesem, natomiast w przeciwnym wypadku zostanie wyrzucony b��d i zwr�ci wartosc 0. Dla komputer�w multiprocesorowych nie ma znaczenia, kt�ry jest u�ywany, mog� jedynie r�ni� si� minimalnie czasy. \end{DoxyCompactList}\item 
int {\bf main} ()
\begin{DoxyCompactList}\small\item\em Funkcja g��wna. W funkcji tej u�ywana jest komunikacja z u�ytkownikiem programu, wykonywane s� funkcje. Wczytany zostaje plik, kt�ry\-: wypisany zostje na ekran, wykonana zostaje p�tla, kt�ra mierzy sredni czas wykonywania algorytmu przez program, nast�pnie u�ytkownik wprowadza drugi plik, kt�ry zostaje por�wnany z wektorem na kt�rym zosta� wykonany algorytm. Na ko�cu zw�rcony zostaje sredni czas wykonywania algorytmu na wektorze w programie. \end{DoxyCompactList}\end{DoxyCompactItemize}


\subsection{Opis szczegółowy}


Definicja w pliku {\bf zad1.\-cpp}.



\subsection{Dokumentacja funkcji}
\index{zad1.\-cpp@{zad1.\-cpp}!algorytm@{algorytm}}
\index{algorytm@{algorytm}!zad1.cpp@{zad1.\-cpp}}
\subsubsection[{algorytm}]{\setlength{\rightskip}{0pt plus 5cm}void algorytm (
\begin{DoxyParamCaption}
\item[{vector$<$ int $>$ \&}]{dane}
\end{DoxyParamCaption}
)}\label{zad1_8cpp_acc25f8b64edccc5c6ccecb552634c616}

\begin{DoxyParams}[1]{Parametry}
\mbox{\tt in}  & {\em dane} & -\/ wektor, na ktorym wykonywany jest algorytm i po wykonaniu algorytmu wektor jest zmieniony, dzi�ki czemu nie musimy tworzy� nowego wektora. \\
\hline
\end{DoxyParams}
\begin{DoxyReturn}{Zwraca}
(brak) 
\end{DoxyReturn}


Definicja w linii 154 pliku zad1.\-cpp.



Oto graf wywoływań tej funkcji\-:


\index{zad1.\-cpp@{zad1.\-cpp}!main@{main}}
\index{main@{main}!zad1.cpp@{zad1.\-cpp}}
\subsubsection[{main}]{\setlength{\rightskip}{0pt plus 5cm}int main (
\begin{DoxyParamCaption}
{}
\end{DoxyParamCaption}
)}\label{zad1_8cpp_ae66f6b31b5ad750f1fe042a706a4e3d4}


Definicja w linii 89 pliku zad1.\-cpp.



Oto graf wywołań dla tej funkcji\-:


\index{zad1.\-cpp@{zad1.\-cpp}!porownanie@{porownanie}}
\index{porownanie@{porownanie}!zad1.cpp@{zad1.\-cpp}}
\subsubsection[{porownanie}]{\setlength{\rightskip}{0pt plus 5cm}void porownanie (
\begin{DoxyParamCaption}
\item[{vector$<$ int $>$ \&}]{dane, }
\item[{vector$<$ int $>$ \&}]{poprawne}
\end{DoxyParamCaption}
)}\label{zad1_8cpp_a2c79249af74d82457d56e9e844c473ed}

\begin{DoxyParams}[1]{Parametry}
\mbox{\tt in}  & {\em dane} & -\/ wektor, kt�ry jest pierwszym wprowadzonyn plikiem przez u�ytkownika i po wykanniu algorytmu. \\
\hline
\mbox{\tt in}  & {\em poprawne} & -\/ wektor, kt�ry wprowadzany jest do programu jako drugi przez u�ytkownika, kt�ry ma by� sprawdzeniem poprawnosci wykonania algorytmu. \\
\hline
\end{DoxyParams}
\begin{DoxyReturn}{Zwraca}
(brak) 
\end{DoxyReturn}


Definicja w linii 160 pliku zad1.\-cpp.



Oto graf wywoływań tej funkcji\-:


\index{zad1.\-cpp@{zad1.\-cpp}!wczytywanie@{wczytywanie}}
\index{wczytywanie@{wczytywanie}!zad1.cpp@{zad1.\-cpp}}
\subsubsection[{wczytywanie}]{\setlength{\rightskip}{0pt plus 5cm}void wczytywanie (
\begin{DoxyParamCaption}
\item[{const string \&}]{tekst, }
\item[{vector$<$ int $>$ \&}]{dane}
\end{DoxyParamCaption}
)}\label{zad1_8cpp_a23c23a4e71065cf19c311b154d5258f6}

\begin{DoxyParams}[1]{Parametry}
\mbox{\tt in}  & {\em tekst} & -\/ zmienna, kt�ra zostaje wprowadzona przez u�ytkownika do programu \\
\hline
\mbox{\tt in}  & {\em dane} & -\/ wektor, do kt�rego zapisywane s� wszystkie elementy pliku. \\
\hline
\end{DoxyParams}
\begin{DoxyReturn}{Zwraca}
(brak) 
\end{DoxyReturn}


Definicja w linii 124 pliku zad1.\-cpp.



Oto graf wywoływań tej funkcji\-:


\index{zad1.\-cpp@{zad1.\-cpp}!wlacz\-Stoper@{wlacz\-Stoper}}
\index{wlacz\-Stoper@{wlacz\-Stoper}!zad1.cpp@{zad1.\-cpp}}
\subsubsection[{wlacz\-Stoper}]{\setlength{\rightskip}{0pt plus 5cm}L\-A\-R\-G\-E\-\_\-\-I\-N\-T\-E\-G\-E\-R wlacz\-Stoper (
\begin{DoxyParamCaption}
{}
\end{DoxyParamCaption}
)}\label{zad1_8cpp_a588eeaaa2b07e0b3d714b779e0b3455f}


Definicja w linii 174 pliku zad1.\-cpp.



Oto graf wywoływań tej funkcji\-:


\index{zad1.\-cpp@{zad1.\-cpp}!wylacz\-Stoper@{wylacz\-Stoper}}
\index{wylacz\-Stoper@{wylacz\-Stoper}!zad1.cpp@{zad1.\-cpp}}
\subsubsection[{wylacz\-Stoper}]{\setlength{\rightskip}{0pt plus 5cm}L\-A\-R\-G\-E\-\_\-\-I\-N\-T\-E\-G\-E\-R wylacz\-Stoper (
\begin{DoxyParamCaption}
{}
\end{DoxyParamCaption}
)}\label{zad1_8cpp_afbe85e6c70cbabd2ed5adabd0ce1b44b}


Definicja w linii 183 pliku zad1.\-cpp.



Oto graf wywoływań tej funkcji\-:


\index{zad1.\-cpp@{zad1.\-cpp}!wypisz@{wypisz}}
\index{wypisz@{wypisz}!zad1.cpp@{zad1.\-cpp}}
\subsubsection[{wypisz}]{\setlength{\rightskip}{0pt plus 5cm}void wypisz (
\begin{DoxyParamCaption}
\item[{vector$<$ int $>$ \&}]{dane}
\end{DoxyParamCaption}
)}\label{zad1_8cpp_af3585718b2c90ae843d952ef08124531}

\begin{DoxyParams}[1]{Parametry}
\mbox{\tt in}  & {\em dane} & -\/ wektor, w kt�rym zapisane s� wszystkie elemnty pliku. \\
\hline
\end{DoxyParams}
\begin{DoxyReturn}{Zwraca}
(brak) 
\end{DoxyReturn}


Definicja w linii 146 pliku zad1.\-cpp.



Oto graf wywoływań tej funkcji\-:



\printindex
\end{document}
